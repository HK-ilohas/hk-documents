\documentclass[12pt,uplatex]{jsreport}
% margin
\usepackage[left=20truemm,right=20truemm,top=20truemm,bottom=13truemm,includefoot]{geometry}
% 行間
\renewcommand{\baselinestretch}{1.5}
\usepackage{setspace}
% 図を現在位置に挿入
\usepackage{here}
% 数式
\usepackage{amsmath,amssymb}
\usepackage{mathtools}
\usepackage{empheq}
\usepackage{amsthm}
\newtheoremstyle{mystyle}%   % スタイル名
    {3pt}%                      % 上部スペース
    {3pt}%                      % 下部スペース
    {\normalfont}%           % 本文フォント
    {}%                      % インデント量
    {\sffamily}%             % 見出しフォント
    {}%                      % 見出し後の句読点, '.'
    {1em}%                     % 見出し後のスペース, ' ' or \newline
    {\thmname{#1}\thmnumber{#2}\thmnote{(#3)}}
% フォント
\usepackage{newtxtext,newtxmath}
% URL
\usepackage{url}
% 目次
\renewcommand{\contentsname}{{\huge 目次}}
\renewcommand{\listfigurename}{{\huge 図目次}}
\renewcommand{\listtablename}{{\huge 表目次}}
\setcounter{tocdepth}{2}
\usepackage[dvipdfmx]{hyperref}
\usepackage{pxjahyper}
% リンク色を全て黒に設定
% 執筆時は何かしらの色を設定したほうが楽
\hypersetup{
    setpagesize=false,
    bookmarksnumbered=true,
    bookmarksopen=true,
    colorlinks=true,
    linkcolor=black,
    citecolor=black,
    urlcolor=black
}

\usepackage{emptypage}
% footer
\usepackage{fancyhdr}
\fancyhf{}
\renewcommand{\headrulewidth}{0pt}

% chapter 
\makeatletter
\renewcommand{\chapter}{%
    \if@openleft\cleardoublepage\else
    \if@openright\cleardoublepage\else\clearpage\fi\fi
    \thispagestyle{fancy} % change fancy from plain
    \vspace*{-5.5em}
    \global\@topnum\z@
    \if@english \@afterindentfalse \else \@afterindenttrue \fi
    \secdef
    {\@omit@numberfalse\@chapter}%
    {\@omit@numbertrue\@schapter}}
\makeatother

\makeatletter
\def\@makechapterhead#1{%
\vspace*{2\Cvs}% 欧文は50pt
{\parindent \z@ \raggedright \normalfont
\ifnum \c@secnumdepth >\m@ne
\huge\headfont \@chapapp\thechapter\@chappos
\vspace*{-1em}
%% \par\nobreak
%% \vskip \Cvs % 欧文は20pt
\fi
\interlinepenalty\@M
\huge \headfont  #1\par\nobreak
\vskip 3\Cvs}} % 欧文は40pt
\makeatother

\begin{document}
    % textlint-disable
    % 表紙の設定
    \thispagestyle{empty}
    \pagenumbering{roman}
    \vspace*{1em}
    \begin{center}
        {\Large 令和5年度\\準学士論文}

        \vspace{4em}

        {\huge \textgt{○○○○○を用いた○○○に関する開発}}

        \vspace{3em}

        {\Large Development of Non-contact Electrocardiogram Measurement Technique by Electrostatic Induction}

        \vspace{17em}

        {\large dXXXXX    高専 太郎}

        {\large 指導教員 ソーシャルデザイン工学科 ○○○○}

        \vspace{5em}

        {\large 2024年1月28日}

        \vspace{2.5em}

        {\large 高知工業高等専門学校 ソーシャルデザイン工学科}
    \end{center}

    % 要旨
    \newpage
    \vspace*{1.6em}
    \begin{center}
        {\large \textgt{要 旨}}

        \vspace{2.5em}

        {\large ○○○○○を用いた○○○に関する開発}

        \vspace{2.5em}

        {\large 高専 太郎}

        \vspace{2em}
    \end{center}

    % 文章
    \begin{spacing}{1.1}
        本研究では,○○○○・・・・・・・・・・・・・・・・・・・・・・・・・・・・・・・・・・・・・・・・・・・・
        ○○○○・・・・・・・・・・・・・・・・・,○○○○・・・・・・・.・・・・・・・・・・○○○○・・・・・・・・・・・・・・・・・○○○○・・・・・・・・・・・・・・・・・○○○○・・・・・・・・・・・・・・・・・○○○○・・・・・・・・・・・・・・・・・○○○○・・・・・・・・・・・・・・・・・○○○○・・・・・・・・・・・・・・・・・○○○○・・・・・・・・・・・・・・・・・○○○○・・・・・・・・・・・・・・・・・○○○○・・・・・・・・・・・・・・・・・○○○○・・・・・・・・・・・・・・・・・○○○○・・・・・・・・・・・・・・・・・○○○○・・・・・・・・・・・・・・・・・○○○○・・・・・・・・・・・・・・・・・○○○○・・・・・・・・・・・・・・・・・○○○○・・・・・・・・・・・・・・・・・○○○○・・・・・・・・・・・・・・・・・○○○○・・・・・・・・・・・・・・・・・○○○○・・・・・・・・・・・・・・・・・○○○○・・・・・・・・・・・・・・・・・○○○○・・・・・・・・・・・・・・・・・○○○○・・・・・・・・・・・・・・・・・○○○○・・・・・・・・・・・・・・・・・.
    \end{spacing}

    \vspace{2em}

    % キーワード
    \noindent
    \textgt{キーワード}
      ○○○○○○,○○○○○○,○○○○○○,○○○○○○

    % Abstract
    \newpage
    \vspace*{1.6em}
    \begin{center}
        {\large \textsf{Abstract}}

        \vspace{2.5em}

        {\large Development of Non-contact Electrocardiogram Measurement Technique by Electrostatic Induction}

        \vspace{2.5em}

        {\large Taro Kosen}

        \vspace{2em}
    \end{center}

    % 文章
    \begin{spacing}{1.1}
        In this study, ○○○○・・・・・・・・・・・・・・・・・・・・・・・・・・・・・・・・・・・・・・・・・・・・
        ○○○○・・・・・・・・・・・・・・・・・,○○○○・・・・・・・.・・・・・・・・・・○○○○・・・・・・・・・・・・・・・・・○○○○・・・・・・・・・・・・・・・・・○○○○・・・・・・・・・・・・・・・・・○○○○・・・・・・・・・・・・・・・・・○○○○・・・・・・・・・・・・・・・・・○○○○・・・・・・・・・・・・・・・・・○○○○・・・・・・・・・・・・・・・・・○○○○・・・・・・・・・・・・・・・・・○○○○・・・・・・・・・・・・・・・・・○○○○・・・・・・・・・・・・・・・・・○○○○・・・・・・・・・・・・・・・・・○○○○・・・・・・・・・・・・・・・・・○○○○・・・・・・・・・・・・・・・・・○○○○・・・・・・・・・・・・・・・・・○○○○・・・・・・・・・・・・・・・・・○○○○・・・・・・・・・・・・・・・・・○○○○・・・・・・・・・・・・・・・・・○○○○・・・・・・・・・・・・・・・・・○○○○・・・・・・・・・・・・・・・・・○○○○・・・・・・・・・・・・・・・・・○○○○・・・・・・・・・・・・・・・・・○○○○・・・・・・・・・・・・・・・・・.
    \end{spacing}

    \vspace{2em}

    % キーワード
    \noindent
    \textit{key words}
     ○○○○○○,○○○○○○,○○○○○○,○○○○○○

    % 目次の出力
    \cfoot{\thepage}
    \tableofcontents
    \listoffigures
    \listoftables
    \clearpage
    % textlint-enable
    \pagenumbering{arabic}
    \pagestyle{fancy}
    \cfoot{- {\thepage} -}

    \chapter{チャプター}
    \section{セクション}
    \subsection{サブセクション}

    本文をここに書く.

    \begin{table}[H]
        \centering
        \caption{$\varepsilon$と$x_1$の値一覧}
        \label{tbl:x1}
        \begin{spacing}{1.1} % 行間を調整する必要あり
        \begin{tabular}{cc}
            \hline
            誤り率$\varepsilon$ & ループ回数の上限$x_1$ \\ \hline
            $0.01$           & 7             \\
            $0.001$          & 10            \\
            $0.0001$         & 14            \\
            $0.00001$        & 17            \\ \hline
        \end{tabular}
        \end{spacing}
    \end{table}

    \chapter*{謝辞}
    \addcontentsline{toc}{chapter}{謝辞}

    イーロン・マスクありがとう.

    % ==bibtex===
    % \bibliographystyle{junsrt}
    % % \fontsize{10pt}{0.5cm}\selectfont{
    % \bibliography{filename}
    % % }
\end{document}